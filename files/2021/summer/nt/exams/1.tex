\documentclass[12pt]{rudin}
\usepackage[]{amsmath}
\usepackage[]{exercise}
\usepackage[]{enumitem}
\usepackage[colorlinks=true, linkcolor=blue, urlcolor=blue]{hyperref}

\newcommand{\floor}[1]{\left\lfloor #1 \right\rfloor}

\renewcommand{\ExerciseName}{Problem}
\renewcommand{\ExerciseHeader}{\noindent\textbf{\ExerciseName\ \ExerciseHeaderNB}\ \ExerciseHeaderTitle\ \ExerciseHeaderOrigin\smallskip}
\renewcommand{\ExerciseHeaderOrigin}{(\ExerciseOrigin)}
\renewcommand{\AnswerHeader}{\bigskip\noindent\textbf{Solution \ExerciseHeaderNB}\newline}

\DeclareMathOperator{\lcm}{lcm}
\let\set\mathbf

\title{Number Theory Midterm I}
\author{RDB}
\date{\today}

\begin{document}

\maketitle

\begin{centering}
    \emph{I feel fine today modulo a slight headache. \\ \hfill --- The
    Hacker's Dictionary}
\end{centering}

\paragraph{INSTRUCTIONS} No outside materials (notes, textbook, internet) or
resources (calculators). Leave your webcam on until you submit and I confirm
that I have your exam.

Good luck!

% 1.
\begin{Exercise}
    Use the Euclidean algorithm to compute the greatest common divisor of 153
    and 64. (5 points)
\end{Exercise}

% 2
\begin{Exercise}
    \begin{enumerate}[label=(\textbf{\alph*})]
        \item State Euclid's lemma. (3 points)
        \item Show that, if a prime $p$ divides $a^2$ for some integer $a$,
            then $p$ divides $a$. (5 points)
    \end{enumerate}
\end{Exercise}

\begin{Answer}
    \begin{enumerate}[label=(\textbf{\alph*})]
        \item If $p | ab$ for a prime $p$ and integers $a$ and $b$, then $p |
            ab$.
        \item If $p | a^2 = a \cdot a$, then $p | a$ or $p | a$. So $p | a$,
            clearly.
    \end{enumerate}
\end{Answer}

% 3.
\begin{Exercise}
    Find the general solution to
    \[
        42x + 35y = 2,
    \]
    if any solutions exist. (10 points)
\end{Exercise}

\begin{Answer}
    The gcd of 42 and 35 is 7, which does not divide 2, so there are no
    solutions.
\end{Answer}

% 4
\begin{Exercise}
    List three solutions to the congruence $2x \equiv 4 \pmod{101}$. Your
    solutions may be congruent mod 101. (10 points)
\end{Exercise}

\begin{Answer}
    An obvious solution is $x = 2$, but there are no others in $\{0, 1, 2,
    \dots, 100\}$, because $\gcd(2, 101) = 1$. We can still get other solutions
    by adding multiples of 101. In particular, $2 + 101 = 103$ and $2 + 202 =
    204$ are also solutions.
\end{Answer}

% 5.
\begin{Exercise}
    Find the general solution to
    \[
        38x + 15y = 1,
    \]
    if any solutions exist. (10 points)
\end{Exercise}

\begin{Answer}
    The gcd of 38 and 15 is 1, so solutions \emph{do} exist. The Euclidean
    algorithm will give $x_0 = 2$ and $y_0 = -5$, so the general solution is
    \[
        x = 2 + 15t; \qquad y = -5 - 38t
    \]
    for an integer $t$.
\end{Answer}

% 6
\begin{Exercise}
    Does the congruence $25 x \equiv 1 \pmod{1000}$ have solutions? If so, how
    many solutions in $\{0, 1, 2, 3, \dots, 999\}$ are there? (10 points)
\end{Exercise}

\begin{Answer}
    No, because $\gcd(25, 1000) = 25$, which does not divide $1$.
\end{Answer}

% 7
\begin{Exercise}
    Show that $\gcd(a, b) = \gcd(a - b, b)$ for any integers $a$ and $b$. (10 points)
\end{Exercise}

\begin{Answer}
    If $d$ is a common divisor of $a$ and $b$, say $a = dk$ and $b = dj$, then
    $a - b = d(k - j)$, so $d$ is a common divisor of $a - b$ and $b$.
    Conversely, if $d$ is a common divisor of $a - b$ and $b$, then it also
    divides $a = (a - b) + b$ by the same argument. Therefore the common
    divisors of $(a, b)$ are the same as the common divisors of $(a - b, b)$,
    and in particular the \emph{greatest} one of them is the same.
\end{Answer}

% 8
\begin{Exercise}
    Show that $2^n \equiv -1 \pmod{3}$ for $n$ odd. (10 points)
\end{Exercise}

\begin{Answer}
    Note that $2 \equiv -1 \pmod{3}$, so $2^n \equiv (-1)^n | \pmod{3}$. If $n$
    is odd $(-1)^n = -1$, so $2^n \equiv -1 \pmod{3}$ for odd $n$.

    Nearly everyone used induction, which was fine, but gross. Here's the best
    induction you could do, I think: If $2^n \equiv -1 \pmod{3}$, then
    \[
        3^{n + 2} \equiv -4 \pmod{3} \equiv -1 \pmod{3}.
    \]
\end{Answer}

% 9
\begin{Exercise}
    Fix integers $a$ and $b$. Suppose that $ax + by = p$ for some integers $x$
    and $y$ and a prime $p$. Prove that $\gcd(a, b)$ is either $1$ or $p$. (10 points)
\end{Exercise}

\begin{Answer}
    By a theorem in class (or writing $a = \gcd(a, b)k$ and $b = \gcd(a, b)j$),
    we see that $\gcd(a, b)$ divides $p$, so it is either $1$ or $p$ since $p$
    is prime.
\end{Answer}

% 10
\begin{Exercise}
    Write $76$ in base $3$. (5 points)
\end{Exercise}

\begin{Answer}
    The largest power to begin with is $3^3 = 27$, and we can fit $2$ of them in:
    \[
        76 - 2 \cdot 3^3 = 22.
    \]
    Repeating:
    \[
        22 - 2 \cdot 3^2 = 4
    \]
    Again:
    \[
        4 - 1 \cdot 3 = 1
    \]
    Finally:
    \[
        1 = 1 \cdot 3^0.
    \]
    So
    \begin{align*}
        76 &= 2 \cdot 3^3 + 2 \cdot 3^2 + 1 \cdot 3^1 + 1 \cdot 3^0 \\
           &= (2211)_3.
    \end{align*}
\end{Answer}

% 11
\begin{Exercise}
    Translate the following numbers from binary into base $10$:
    \begin{enumerate}[label=(\textbf{\alph*})]
        \item $(1)_2$ (1 point)
        \item $(11)_2$ (1 point)
        \item $(111)_2$ (1 point)
        \item $(1111)_2$ (1 point)
        \item Prove, by induction, that
        \[
            (\underbrace{11\cdots 1}_{n})_2 = 2^n - 1
        \]
        for all positive integers $n$. (5 points)
    \end{enumerate}
\end{Exercise}

% 12
\begin{Exercise}
    Write $(1024)_5$ in base 10. (5 points)
\end{Exercise}

\begin{Answer}
    \begin{align*}
        (1024)_5 &= 1 \cdot 5^3 + 0 \cdot 5^2 + 2 \cdot 5^1 + 4 \cdot 5^0 \\
                 &= 125 + 10 + 4 \\
                 &= 139.
    \end{align*}
\end{Answer}

\paragraph{BONUS PROBLEM} The exam is graded out of 102 points, all accounted
for in the previous problems. The following problem is worth an additional 10
points.

\begin{Exercise}
    \begin{enumerate}[label=(\textbf{\alph*})]
        \item Find the general solution to the diophantine equation
            \begin{equation*}
                10x + 11y = 200.
            \end{equation*}
            (2 points)

        \item Show that there exists only a single solution with $x$ and $y$
            both positive. (5 points)

        \item Find that positive solution. (3 points)
    \end{enumerate}
\end{Exercise}

\end{document}
