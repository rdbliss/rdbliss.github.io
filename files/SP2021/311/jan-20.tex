\documentclass[12pt]{article}
\usepackage[noanswer]{exercise}
\usepackage[]{amsthm}
\usepackage[]{amsmath}
\usepackage[]{hyperref}

\newtheorem{theorem}{Theorem}
\newtheorem{lemma}{Lemma}
\newtheorem{proposition}{Proposition}
\newtheorem{corollary}{Corollary}

\theoremstyle{definition}
\newtheorem{definition}{Definition}

% I don't like how the "exercise" package normally lays things out.
% Feel free to copy this if you agree.
\renewcommand{\ExerciseHeader}{\noindent\textbf{\ExerciseName\ \ExerciseHeaderNB}\ \ExerciseHeaderTitle\ \ExerciseHeaderOrigin\smallskip}
\renewcommand{\ExerciseHeaderOrigin}{(\ExerciseOrigin)}
\renewcommand{\AnswerHeader}{\bigskip\noindent\textbf{Solution \ExerciseHeaderNB}\newline}

\title{Recitation Notes}
\author{Robert Dougherty-Bliss}
\date{20 January 2021}

\begin{document}

\maketitle

Welcome to real analysis!

Today marks the first step of your journey through one of the principle pillars
of mathematics: \emph{analysis}. The field is so vast that I could not possibly
summarize it in a few words. But if I had to, I would say that analysis is the
rigorous study of \emph{change} and \emph{estimates}.

% Adding the * gets rid of the section numbering.
\section*{Logistics}%
\label{sec:logistics}

\begin{enumerate}
    \item I am Robert Dougherty-Bliss, your TA for this semester. For Professor
        Coley, I run \emph{workshops} every week. For now, this is 15-30
        minutes of me going over material, and then an hour or so of answering
        your questions.

        For Professor Klatt, my rough plan is to record the weekly workshops I
        do with the other section and post them online for you to watch. Maybe
        this will change.

    \item We have a Canvas page. The syllabus is on the Canvas page.

    \item Textbook: \url{https://elliespathtostatistics.files.wordpress.com/2018/03/abbott-second-edition.pdf}

    \item Office hour survey: \url{https://www.when2meet.com/?10713170-4UARI}

    \item Join the Rutgers Math Discord! \url{https://discord.gg/rutgersmath}

        I usually hang out there and answer questions.
\end{enumerate}

\section*{New stuff}%
\label{sec:new_stuff}

Like you should for any journey, we will begin today by preparing our
equipment. In this case, our equipment consists of \emph{methods of proof}. I
cannot say, ``in order to prove X, do Y,'' but I can lay out some rough logical
tools. These tools are:

\begin{enumerate}
    \item Direct proof.
    \item Proof by contrapositive.
    \item Proof by contradiction.
    \item Proof by induction.
\end{enumerate}

In theory you've seen these before, but I'm sure that you're rusty. Let's see
some examples.

\subsection*{Direct proof}

The first and most ``obvious'' way to prove something is to do it directly. You
take the assumptions given to you and move them around until the conclusion
falls out.

\begin{definition}
    An integer $n$ is \emph{even} provided that $n = 2k$ for some integer $k$.
    On the other hand, $n$ is \emph{odd} provided that $n = 2m + 1$ for some
    integer $m$.
\end{definition}

\begin{proposition}
    The sum of two even integers is even. (If $x$ and $y$ are even integers,
    then $x + y$ is even.)
\end{proposition}

\begin{proof}
    Let $x$ and $y$ be even integers. That is, $x = 2n$ and $y = 2m$ for some
    integers $n$ and $m$. Then
    \begin{equation*}
        x + y = 2(n + m),
    \end{equation*}
    and since $n + m$ is an integer, it follows that $x + y$ is even.
\end{proof}

\begin{proposition}
    The product of two odd integers is odd. (If $x$ and $y$ are odd integers,
    then $xy$ is odd.)
\end{proposition}

\begin{proof}
    Let $x = 2n + 1$ and $y = 2m + 1$ for integers $n$ and $m$. Then
    \begin{equation*}
        xy = (2n + 1)(2m + 1) = 2(2nm + n + m) + 1,
    \end{equation*}
    so $xy$ is odd.
\end{proof}

\begin{Exercise}
    Prove that the sum of two odd integers is even.
\end{Exercise}

\subsection*{Contrapositive}%

The statement ``if $P$, then $Q$'' is \emph{logically equivalent} to ``if not
$Q$, then not $P$.'' The latter statement is called the \emph{contrapositive}
of the former, and proving either proves the other. In some cases the
contrapositive is much easier to prove than the original statement. Other times
there isn't much of a difference.

\begin{proposition}
    If $n^2$ is even, then $n$ is even.
\end{proposition}

Writing $n^2 = 2k$ does not seem very helpful. Instead, we'll try the
contrapositive.

\begin{proof}
    Suppose that $n$ is odd, so that $n = 2m + 1$ for some integer $m$. Then
    $n^2 = 2(2m^2+ 2m) + 1$ is odd, so our original claim is true.
\end{proof}

Note that the contrapositive is \emph{not} the same as the \emph{converse}. The
converse of ``if $P$, then $Q$'' is ``if $Q$, then $P$.'' In general these are
completely different statements. If both happen to be true, then we say ``$P$
if and only if $Q$,'' sometimes shortened ``$P$ iff $Q$.''

\begin{Exercise}
    Is the following true?
    \begin{quote}
        If $x$ and $y$ are integers, then $x + y$ is even iff $x$ and $y$ are
        both even.
    \end{quote}
\end{Exercise}

\begin{Exercise}
    Show that $n^2$ is odd iff $n$ is odd. (Hint: Do one half directly, and one
    half with the contrapositive.)
\end{Exercise}

\subsection*{Contradiction}%

If a statement cannot be false, then it must be true. (This is controversial.)
So, to prove that something is true, start by assuming that it is \emph{false},
and try to show that this is impossible.

Before I give you an example, I should say that not many things \emph{require}
a proof by contradiction. In fact, most proofs by contradiction are actually
proofs by contrapositive in disguise! I will yell at you if you make this
mistake.

Now for the most famous example where contradiction is really necessary.

\begin{proposition}
    $\sqrt{2}$ is irrational.
\end{proposition}

\begin{proof}
    Suppose, to the contrary, that $\sqrt{2} = a / b$ for coprime integers $a$
    and $b$, i.e., $a$ and $b$ do not share a common factor other than $1$.
    Then
    \begin{equation*}
        2 = \frac{a^2}{b^2},
    \end{equation*}
    or $2 b^2 = a^2$. It follows that $a^2$ is even, and so $a$ is as well,
    meaning that $a = 2a'$ for some integer $a'$. Our equation becomes
    \begin{equation*}
        2 b^2 = (2a')^2 = 4a'^2,
    \end{equation*}
    so $b^2 = 2a'^2$. From this we see that $b^2$ is even, and therefore $b$ is
    as well. But now $2$ divides both $a$ and $b$, which contradicts that they
    share no common divisors but $1$. Therefore $\sqrt{2}$ \emph{cannot} be
    rational.
\end{proof}

\begin{Exercise}
    Mimic the above proof to show that $2^{1/3}$ is irrational. (Hint: If $2$
    divides $xy$, then it divides either $x$ or $y$.) Now show that $2^{1/n}$
    is irrational for all $n \geq 2$.
\end{Exercise}

\subsection*{Induction}

Induction is a tool for when you want to prove something about all positive
integers, and somehow each integer with the property implies that the next
integer has it too.

The general setup is this: Suppose that $P(n)$ is some statement about the
integer $n$, maybe $P(n) = n^2 \geq n$. If $P(1)$ is true, and $P(n)$ implies
$P(n + 1)$ for every $n \geq 1$, then $P(n)$ is true for all $n \geq 1$. (You
can replace $1$ with any integer.)

\begin{proposition}
    For all $n \geq 4$,
    \begin{equation*}
        n! \geq 2^n.
    \end{equation*}
\end{proposition}

\begin{proof}
    Suppose that $n! \geq 2^n$ for some $n \geq 4$. Then,
    \begin{equation*}
        \frac{(n + 1)!}{2^{n + 1}} = \frac{n + 1}{2} \frac{n!}{2^n} \geq \frac{n + 1}{2}.
    \end{equation*}
    Since $n \geq 4$ we get $(n + 1) / 2 \geq 1$, which implies
    \begin{equation*}
        \frac{(n + 1)!}{2^n} \geq 1.
    \end{equation*}
    At $n = 4$, we have
    \begin{equation*}
        4! = 24 \geq 16 = 2^4.
    \end{equation*}
    Thus, by induction, $n! \geq 2^n$ for all $n \geq 4$.
\end{proof}

Here's a statement with a true induction step, but is false for all nonnegative
integers: $n! = 0$ for all $n \geq 0$.

\begin{Exercise}
    The \emph{Fibonacci} numbers are defined recursively as follows:
    \begin{align*}
        F_0 &= 0 \\
        F_1 &= 1 \\
        F_{n + 2} &= F_{n + 1} + F_n.
    \end{align*}
    Prove via induction that
    \begin{equation*}
        \sum_{k = 0}^n F_k = F_{n + 2} - 1
    \end{equation*}
    for all integers $n \geq 0$.

    (Hint: $\sum_{k = 0}^{n + 1} F_k = \sum_{k = 0}^n F_k + F_{n + 1}$.)
\end{Exercise}

\begin{Answer}
    For $n = 0$, the identity reads
    \begin{equation*}
        0 = \sum_{k = 0}^n F_k = F_2 - 1 = 0,
    \end{equation*}
    which is true. Now, suppose that
    \begin{equation*}
        \sum_{k = 0}^n F_k = F_{n + 2} - 1
    \end{equation*}
    for some $n \geq 0$. Then,
    \begin{align*}
        \sum_{k = 0}^{n + 1} F_k &= \sum_{k = 0}^n F_k + F_{n + 1} \\
                                 &= F_{n + 2} + F_{n + 1} - 1 \\
                                 &= F_{n + 3} - 1 \\
                                 &= F_{(n + 1) + 2} - 1.
    \end{align*}
    Therefore, by induction, the identity holds for all $n \geq 0$.

    (Method two, which you are not allowed to use: Check it with a computer for
    $n \leq 100$.)
\end{Answer}

\section*{A note on style}%

Style and taste is important in all writing, even in mathematics. I'm not going
to explain all of the finer points, but I will point you to this document,
written in part by mathematical wiz and computer science guru Donald Knuth:

\url{https://jmlr.csail.mit.edu/reviewing-papers/knuth_mathematical_writing.pdf}

Read the first chapter when you have time! You'll be glad that you did.

\section*{A note on \LaTeX}%

LaTeX, styled \LaTeX, is a ``typesetting'' system widly used in mathematics.
\LaTeX\ turns formatted plaintext documents (created using a text editor such
as Notepad++) into nicely rendered PDF files. It has \emph{very} good support
for mathematical symbols and a number of other niche topics. Virtually all
mathematicians use \LaTeX\ to write papers, prepare lecture notes, and so on.
In fact, I'm using it to write these notes right now.

It is encouraged---but \emph{not} required---that you write up your homework,
midterm projects, etc.~in \LaTeX. This makes it easier to read and will make
you feel fancy.

Some of you have seen \LaTeX\ and some of you have not. There are a number of
resources online that will teach you how to prepare and ``compile'' a \LaTeX
document. To give you examples of good \LaTeX\ documents, I will occasionally
prepare my workshop notes in \LaTeX\ and post them online somewhere every week.
(I can't promise I will do this all the time. It takes more time!)

\end{document}
