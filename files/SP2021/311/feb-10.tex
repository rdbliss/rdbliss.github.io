\documentclass[12pt]{article}
\usepackage[noanswer]{exercise}
\usepackage[]{amsthm}
\usepackage[]{amsmath}
\usepackage[]{enumitem}
\usepackage[]{hyperref}

\newtheorem{theorem}{Theorem}
\newtheorem{lemma}{Lemma}
\newtheorem{proposition}{Proposition}
\newtheorem{corollary}{Corollary}

\theoremstyle{definition}
\newtheorem{definition}{Definition}

\let\set\mathbf

% I don't like how the "exercise" package normally lays things out.
% Feel free to copy this if you agree.
\renewcommand{\ExerciseHeader}{\noindent\textbf{\ExerciseName\ \ExerciseHeaderNB}\ \ExerciseHeaderTitle\ \ExerciseHeaderOrigin\smallskip}
\renewcommand{\ExerciseHeaderOrigin}{(\ExerciseOrigin)}
\renewcommand{\AnswerHeader}{\bigskip\noindent\textbf{Solution \ExerciseHeaderNB}\newline}

\title{Workshop Notes}
\author{Robert Dougherty-Bliss}
\date{10 February 2021}

\begin{document}

\maketitle

% Adding the * gets rid of the section numbering.
\section*{Logistics}%
\label{sec:logistics}

Nothing to say here.

\section*{New stuff}%
\label{sec:new_stuff}

Since I last saw you, you learned more about sequences and got some new
homework. We're going to talk about some sequence topics and a problem from the
new homework, as well as any other questions you might have.

\subsection*{Ces\`aro means}%
\label{sub:cesaro_means}

If we have a sequence $x(n)$, then we can take \emph{averages} of it by
defining
\begin{equation*}
    y(n) = \frac{1}{n} \sum_{k = 1}^n x(k).
\end{equation*}

\textbf{Plausible statement:} If $x(n)$ converges, then the averages also
converge to the same thing.

This is true. It's called Ces\`aro summation. The proof illustrates one of the
main techniques of analysis: Splitting a problem into two managable parts.

\begin{theorem}
    If $x(n) \to x$, then
    \begin{equation*}
        y(n) = \frac{1}{n} \sum_{k = 1}^n x(k) \to x.
    \end{equation*}
\end{theorem}

\begin{proof}
    We want to show that, given any $\epsilon > 0$, there exists some integer
    $N$ such that
    \begin{equation*}
        |y(n) - x| < \epsilon
    \end{equation*}
    for $n > N$.

    The trick about averaging is that we can break up a single $x$ into many
    terms:
    \begin{equation*}
        y(n) - x = \frac{1}{n} \sum_{k = 1}^n x(k) - x = \frac{1}{n} \sum_{k = 1}^n (x(k) - x).
    \end{equation*}
    Therefore, by the triangle inequality,
    \begin{equation*}
        |y(n) - x| \leq \frac{1}{n} \sum_{k = 1}^n |x(k) - x|.
    \end{equation*}
    For ``large'' $k$, we have the ``sharp'' bound $|x(k) - k| < \epsilon$. For
    ``small'' $k$, we only have the ``coarse'' fact that $|x(k) - k|$ is
    bounded, since $x(k) - k \to 0$.

    To make this idea rigorous, pick $M$ such that $|x(k) - x| \leq M$ for all
    $k$ (by boundedness), and $N$ such that $|x(k) - x| < \epsilon$ for $k > N$
    (by convergence). Then, for $n > N$,
    \begin{align*}
        |y(n) - x|
            &\leq \frac{1}{n} \sum_{k = 1}^N |x(k) - x| + \frac{1}{n} \sum_{N < k \leq n} |x(k) - x| \\
            &\leq \frac{NM}{n} + \frac{\epsilon (n - N)}{n} \\
            &\leq \frac{NM}{n} + \epsilon.
    \end{align*}
    To get the right bound here, we need to turn that $NM / n$ into something
    involving $\epsilon$. But $N$ and $M$ are both fixed relative to
    $\epsilon$, so we can just require $n$ to be really big in comparison. That
    is, let $N'$ be such that $N' > N$ and $NM / n < \epsilon$ for $n > N'$.
    Then, for $n > N'$, the bound becomes
    \begin{equation*}
        |y(n) - x| \leq \epsilon + \epsilon = 2\epsilon.
    \end{equation*}
    Since $\epsilon$ was arbitrary, this is equivalent and shows that $y(n) \to
    x$.
\end{proof}

\textbf{Interesting question:} If the averages converge, does $x(n)$ converge?

\subsection*{Limits superior and inferior}%
\label{sub:limits_superior_and_inferior}

Not every sequence has a limit, but sometimes we want to talk about ``limits''
even when they don't exist. For example, $a(n) = (-1)^n$ does not ``converge,''
but it kind of has two ``pseudo-limits,'' $1$ and $-1$. These ``peusdo-limits''
are made precise by the \emph{limits superior and inferior}.

Given any sequence $a(n)$, there exists a number $\beta$ (possibly $\infty$)
such that:
\begin{enumerate}
    \item For every $\epsilon > 0$, we have $a(n) < \beta + \epsilon$ eventually.

    \item For every $\epsilon > 0$, there are infinitely many $n$ such that
        $a(n)$ is in $(\beta - \epsilon, \beta]$.
\end{enumerate}
The number $\beta$ is called the \emph{limit superior} of $a(n)$, and we denote
it by $\beta = \limsup_n a(n)$.

The above properties actually uniquely define the limit superior, but they
don't prove that it exists. Let's go through how Abbott does it. For now,
suppose that $a(n)$ is a bounded sequence.

\begin{enumerate}[label=(\alph*)]
    \item Prove that $y(n) = \sup \{a(k) \mid k \geq n\}$ converges. Let
        $\limsup_n a(n) = \lim_n y(n)$.

        % The sets are decreasing.

    \item Give the analogous definition for $\liminf_n a(n)$.

        % Switch to infs. Now the sets are increasing.

    \item Show that $\liminf_n a(n) \leq \limsup_n a(n)$. [Hint: If $x(n)$ and
        $y(n)$ converge and $x(n) \leq y(n)$, then $\lim_n x(n) \leq \lim_n
        y(n)$. (You should prove this!)]

        % The hint gives it away.

    \item Show that $\lim_n a(n) = x$ iff $\liminf_n a(n) = x = \limsup_n
        a(n)$. (You can't use the above properties I mentioned unless you prove
        them!)

        % Say that a(n) -> x. For every eps > 0, we have a(n) in (x - eps, x + eps)
        % for large n. Thus sup {a(k) | k >= n} <= x + eps for large n, which gives
        % limsup a(n) <= x + eps for every eps > 0. Thus limsup a(n) <= x.
        % Same argument with flipped signs works for liminf.

        % Conversely, say that limsup = liminf = x. For every eps > 0, we have
        % sup {a(k) | k >= n} < x + eps eventually, so a(k) <= x + eps eventually.
        % By an analogous argument, a(k) >= x - eps eventually, so a(k) is in
        % [x - epsilon, x + epsilon] eventually.
\end{enumerate}

\begin{Exercise}
    If $a(n) \leq b(n)$, we can't always say that $\lim_n a(n) \leq \lim_n
    b(n)$, because the limits might not even exist. Show that
    \begin{equation*}
        \limsup_n a(n) \leq \limsup_n b(n)
    \end{equation*}
    and
    \begin{equation*}
        \liminf_n a(n) \leq \liminf_n b(n)
    \end{equation*}
    whenever $a(n) \leq b(n)$ for all $n$.
\end{Exercise}

\end{document}
