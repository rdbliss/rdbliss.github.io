\documentclass[12pt]{article}
\usepackage[noanswer]{exercise}
\usepackage[]{amsthm}
\usepackage[]{amsmath}
\usepackage[]{enumitem}
\usepackage[]{hyperref}

\newtheorem{theorem}{Theorem}
\newtheorem{lemma}{Lemma}
\newtheorem{proposition}{Proposition}
\newtheorem{corollary}{Corollary}

\theoremstyle{definition}
\newtheorem{definition}{Definition}

\let\set\mathbf

% I don't like how the "exercise" package normally lays things out.
% Feel free to copy this if you agree.
\renewcommand{\ExerciseHeader}{\noindent\textbf{\ExerciseName\ \ExerciseHeaderNB}\ \ExerciseHeaderTitle\ \ExerciseHeaderOrigin\smallskip}
\renewcommand{\ExerciseHeaderOrigin}{(\ExerciseOrigin)}
\renewcommand{\AnswerHeader}{\bigskip\noindent\textbf{Solution \ExerciseHeaderNB}\newline}

\title{Workshop Notes}
\author{Robert Dougherty-Bliss}
\date{3 February 2021}

\begin{document}

\maketitle

% Adding the * gets rid of the section numbering.
\section*{Logistics}%
\label{sec:logistics}

Nothing to say here.

\section*{New stuff}%
\label{sec:new_stuff}

Since I last saw you, you learned about cardinality and sequences. Sequences
will get properly confusing when we get to some theorems, but for now I imagine
thatt cardinality is very weird for everyone. Let's remind ourselves of what's
going on.

One way to measure a set is to just count all the things in it;
\begin{equation*}
    |\{a, b, c\}| = 3.
\end{equation*}
Easy. Another way is to ``line up'' one set with another. For example:
\begin{align*}
    &\{a, b, c\} \\
    &\{1, 2, 3\}
\end{align*}
These sets are the same size because we can line them up. If we agree that
$\{1, 2, 3\}$ has size $3$, then $\{a, b, c\}$ has size $3$ as well.

This second way seems like a really stupid way to measure sets. Until you learn
about infinite sets. You can't ``count'' infinite sets. It doesn't make any
sense. You could say that every infinite set has size $\infty$, but you're
going to miss a \emph{weird} thing that happens: infinite sets can have
different sizes if you measure using the second way.

Formally, we say that two sets $A$ and $B$ have the same cardinality if there
exists a bijection between them (a way to line them up). That is, there's a
function $f \colon A \to B$ such that:
\begin{enumerate}
    \item For all $b \in B$, there is some $a \in A$ such that $f(a) = b$ (surjectivity); and
    \item If $f(a_1) = f(a_2)$, then $a_1 = a_2$ (injectivity).
\end{enumerate}

For example, $\{1, 2, 3, \dots\}$ has the same size as $\{2, 3, 4, \dots\}$,
because we can define
\begin{equation*}
    f(n) = n + 1,
\end{equation*}
which is a bijection. (Exercise!)

More interestingly, $\{1, 2, 3, \dots \}$ has the same size as $\{2, 4, 6, 8,
\dots\}$. (Define $f(n) = 2n$.) That is, the even natural numbers have the same
size as all the natural numbers themselves! The odds \emph{also} have the same
size as the natural numbers. (Define $f(n) = 2n + 1$.) If we say that
$\aleph_0$ is the ``size'' of $\{1, 2, 3, \dots\}$ then we have an equation
like
\begin{equation*}
    \aleph_0 + \aleph_0 = \aleph_0.
\end{equation*}
That's weird. It's not so weird if we think of $\aleph_0$ as $\infty$, but
there are \emph{other} infinities! Famously, $\mathbf{R}$ is
\emph{uncountable}, in the sense that there is no bijection from $\mathbf{N}$
to $\mathbf{R}$. (See Abbott.) But clearly $\mathbf{R}$ is bigger than
$\mathbf{N}$\dots\ this is all very weird.

I won't say anything more---that's what Abbott is for---but I'm sure that you
have lots of questions about cardinality and homework. Fire away!

% Hidden secrets.

% I want everyone to have their own questions. If they don't, then I'm going to
% walk us through a few exercises.

% Algebraic numbers are countable.

% 1.5.5.
% 1.5.6. Hard!
% 1.5.8. Very hard!

% If lim_n (a(n) + b(n)) = 0, does lim_n a(n) = - lim_n b(n)?

\subsection*{Promises}%
\label{sub:promises}

I forgot to prove something for you that I promised I would. To make it up to
you, I'll show you something fancier to whet your appetite for things to come.
(I won't talk about this at all during the workshop. This is just a bonus read
if you're interested.)

\begin{theorem}
    If $S$ is a nonempty, bounded set of integers, then $\sup S$ is contained
    in $S$.
\end{theorem}

This is actually a special case of a more general result. First, I need to
define a new object.

\begin{definition}
    A point $x$ is a \emph{limit point} of a set $S$ if every
    $\epsilon$-neighborhood contains a point of $S$ not equal to $x$.
\end{definition}

For example, $0$ is a limit point of $\{1, 1 / 2, 1 / 3, 1 / 4, \dots\}$.

A nice fact about sups and infs is that they're either in the underlying set or
limit points of it.

\begin{proposition}
    If $\sup S \notin S$, then $\sup S$ is a limit point of $S$.
\end{proposition}

\begin{proof}
    For every $\epsilon > 0$, we can find some $s \in S$ such that $\sup S -
    \epsilon < s < \sup S$, the last inequality being strict since $\sup S
    \notin S$. Thus every $\epsilon$-neighborhood of $\sup S$ contains a point
    of $S$ not equal to $\sup S$, so $\sup S$ is a limit point of $S$.
\end{proof}

\begin{corollary}
    Let $S$ be a nonempty set, bounded from above, without limit points. Then
    $\sup S \in S$.
\end{corollary}

\begin{proof}
    Since $S$ has no limit points, we must have $\sup S \in S$ by the previous
    proposition.
\end{proof}

Now Theorem~1 is easy!

\begin{proof}[Proof of Theorem 1]
    If $S$ is a set of integers, then $S$ has no limit points. (Every element
    is at least $1$ away from every other element.) Thus, if $\sup S$ exists,
    $\sup S \in S$ by the previous corollary.
\end{proof}

All of this is the application of general \emph{metric space theory} to
$\set{R}$. It turns out that a lot of arguments we make about sequences, sets,
and so on, generalize to other situations without any modifications. If this
sounds interesting, go look up ``metric spaces.''

\end{document}
