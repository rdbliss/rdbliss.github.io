\documentclass[12pt]{article}
\usepackage[noanswer]{exercise}
\usepackage[]{amsthm}
\usepackage[]{amsmath}
\usepackage[]{enumitem}
\usepackage[]{hyperref}

\newtheorem{theorem}{Theorem}
\newtheorem{lemma}{Lemma}
\newtheorem{proposition}{Proposition}
\newtheorem{corollary}{Corollary}

\theoremstyle{definition}
\newtheorem{definition}{Definition}

\let\set\mathbf

% I don't like how the "exercise" package normally lays things out.
% Feel free to copy this if you agree.
\renewcommand{\ExerciseHeader}{\noindent\textbf{\ExerciseName\ \ExerciseHeaderNB}\ \ExerciseHeaderTitle\ \ExerciseHeaderOrigin\smallskip}
\renewcommand{\ExerciseHeaderOrigin}{(\ExerciseOrigin)}
\renewcommand{\AnswerHeader}{\bigskip\noindent\textbf{Solution \ExerciseHeaderNB}\newline}

\title{Workshop Notes}
\author{Robert Dougherty-Bliss}
\date{27 January 2021}

\begin{document}

\maketitle

% Adding the * gets rid of the section numbering.
\section*{Logistics}%
\label{sec:logistics}

Nothing to say here.

\section*{New stuff}%
\label{sec:new_stuff}

This week you learned about about the real numbers. You already knew most of
the algebraic structure of the reals (addition, multiplication, inequalities,
etc.), but the real kicker is the \emph{completeness axiom}.

% There's a "\ " after \sup because \sup(0, 1) looks weird.
% Normally you would just do \sup S for a set S, with no "\ ".
% The "\ " adds a little bit of space.
Consider the set $(0, 1)$ of all reals between $0$ and $1$. Note that $x \leq
2$ for every $x \in (0, 1)$. That is, $2$ is an \emph{upper bound} for $(0,
1)$. There are many such upper bounds: $1 + 1/2$, and $1 + 1/3$, and $1
+ 1/999999999$, and so on. The completeness axiom tells us that there is a
\emph{least} one, i.e., an upper bound which we cannot ``improve'' by making it
smaller. We call this least upper bound the \emph{supremum} of $(0, 1)$, and
denote it $\sup\ (0, 1)$. Here, $\sup\ (0, 1) = 1$.

The axiom of completeness is silly for $(0, 1)$---obviously the least upper
bound exists in this case---but it is \emph{crucial} in cases where the least
upper bound is not obvious. I don't want to spoil Professor Coley's thunder, so
I'll just say that the completeness axiom is the most important property of the
reals.

The axiom is also a good philosophical example of \emph{infinity}. If there
were only finitely many upper bounds of $(0, 1)$, then we could go through the
list and pick the smallest. Since there are infinitely many we can't do this
``by hand,'' but the axiom says that we can do it abstractly. Trying to pick
the smallest element of an infinite set is a recurring problem throughout
analysis.

Now I'm going to open the floor to questions and discussion from everyone. If
discussion flows, perfect. Otherwise, I selected a few exercises.

\begin{Exercise}
    \begin{enumerate}[label=(\alph*)]
        \item What is $\sup \set{R}$?

        \item Can you construct a set $S$ with $\inf S \geq \sup S$? If yes,
            what are \emph{all} such sets? If no, prove that it is impossible.

        \item Can you construct a nonempty set, bounded from above, that does
            not contain its supremum?

        \item Can you construct a nonempty set of \emph{integers}, bounded from
            above, that does not contain its supremum?

        \item Can you construct a nonempty set bounded from above where the
            supremum is not obvious?
    \end{enumerate}
\end{Exercise}

\begin{Exercise}
    (Abbott, 1.4.8)

    The nested interval theorem says that $\bigcap_{k \geq 1} I_k$ is nonempty
    if $\{I_k\}$ is a sequence of nonempty, bounded, nested, closed intervals.
    Do we need all of those conditions? That is, what happens if the intervals
    are open? Or not nested? Or unbounded? The following questions explore some
    of these ideas.

    Give an example of each of the following, or argue that it is impossible.

    \begin{enumerate}[label=(\alph*)]
        \item \emph{Disjoint}, nonempty, bounded from above sets $A$ and $B$
            such that $\sup A = \sup B$, $\sup A \notin A$, and $\sup B \notin
            B$.

        \item A sequence of nested, \emph{open} intervals $I_1 \supseteq I_2
            \subseteq \cdots$ with $\bigcap_{k \geq 1} I_k$ having a nonzero
            but \emph{finite} number of elements.

        \item A sequence of nested, \emph{unbounded}, closed intervals $L_1
            \supseteq L_2 \supseteq \cdots$ with $\bigcap_{k \geq 1} L_k$
            empty.

        \item A sequence of closed, bounded intervals $J_1, J_2, \dots$ such
            that $\bigcap_{k = 1}^n J_k$ is nonempty for every positive integer
            $n$, yet $\bigcap_{k \geq 1} J_k$ is empty.
    \end{enumerate}
\end{Exercise}

\begin{Exercise}
    (Abbott, 1.3.4.) [I don't expect that we'll have time to get to this, so
    I'm not going to bother typing it up.]
\end{Exercise}

% Hidden secrets.

% I want everyone to have their own questions. If they don't, then I'm going to
% walk us through a few exercises.

% Describe a set without a supremum. The same for infimum.

% 1.3.2.
% 1.3.4. How does the supremum behave under unions?
% 1.4.8.

\end{document}
